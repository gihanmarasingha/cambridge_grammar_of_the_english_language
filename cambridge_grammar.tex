\documentclass{scrarticle}

\usepackage{qtree}

\begin{document}

\title{Notes on the Cambridge Grammar of the English Language} \author{Gihan Marasingha}

\maketitle

\section{About this document}

These are my notes on \emph{The Cambridge Grammar of the English Language} (CGEL).

\section{Basic concepts}

\subsection{Grammar as a description of English}

A grammar of a language is one that aims to ‘describe the principles or rules governing the form and
meaning of words, phrases, clauses and sentences’. It has two parts \textbf{syntax} and
\textbf{morphology}. The former being concerned with the way in which words form larger units
(phrases, clauses, sentences) and the latter with the construction of words themselves.

CGEL also touches on some topics outside syntax and morphology. It examines \textbf{semantics},
distinguishing between \textbf{grammatical semantics}, this being meaning that follows simply from
the language system) and \textbf{pragmatics}, this being about meaning in particular situations.

Unlike mathematical definitions, those in CGEL are not stipulative. Rather, they are descriptive.
They must be, for the grammar contains no undefined terms.


Thus, where CGEL defines the grammatical category \emph{noun}, it does so by describing a set of
characteristics of \textbf{prototypical members} (also called the \textbf{central} or \textbf{core}
members) of the category. This is to acknowledge that some words that we class as nouns do not
satisfy all the attributes specified in the grammar.

For example, \emph{grape} is a prototypical noun: it denotes a physical object, it has an
inflectional form contrast between singular and plural, it functions as the head of a noun phrase.
However, the noun \emph{equipment} shows no such inflectional form.

Likewise, in ‘race discrimination’, the noun race does not act as head of a noun phrase. Instead, it
is a modifier of the noun ‘discrimination’. However, this does not make ‘race’ an adjective. The
word race does not satisfy many of the other characteristics of adjective. For example, it cannot be
used predicatively: ‘I am race’ is ungrammatical.

Put another way, the grammar is not sufficient, by itself, to determine what is and what is not a
noun. This discrimination can only be established by something external: the observation of how we
use particular words or phrases in practice.

My thought on the matter is that a grammar is simply descriptive in the same sense that Linnaeus’
taxonomy is descriptive. In each case, there is something ‘out there’ that we are attempting to
codify. But the thing itself (be it the relationships between biological species or the construction
of English language sentences) has no intrinsic \emph{formal} structure. There will always be ‘edge
cases’ that are difficult to pin down.


\section{The components of the grammar}

As mentioned, the main components of the grammar are the study of syntax and the study of
morphology. We briefly treat each of these in turn.

\subsection{Syntax}

The theoretical framework for syntax in CGEL is founded on three principles:

\begin{enumerate}
    \item Sentences have parts, which may themselves be parts.
    \item The parts of sentences belong to a limited ranges of types.
    \item The parts have specific roles or functions within the larger parts they belong to.
\end{enumerate}

These three principles correspond and lead to the notions of \textbf{constituent structure} analysis,
\textbf{syntactic categories} and \textbf{grammatical functions}.

\subsubsection{Constituent structure}

Consider the foundational notion that a sentence can be divided into parts, each of which may also
be divided into parts, called \textbf{constituents}.

Figure \ref{fig:abirdhitthecar} shows one way to decompose the clause, `a bird hit the car' as a
hierarchy of constituents.

\begin{figure}[ht]
\Tree [.{} [.{} [.{} a ] [.{} bird ]  ] [.{} [.{} [.{} hit ] ] [.{} [.{} the ] [.{} car ]]  ] ]
\caption{A syntax tree for a simple sentence}
\label{fig:abirdhitthecar}
\end{figure}

In this diagram, the phrases `a bit' and `hit the car' are the \textbf{immediate constitutions} and
the leaf nodes (here words) are the \textbf{ultimate constituents} of the sentence.

Evidently, this is not the only possible way to impose a hierarchy on (equally to \emph{analyse}) `a
bird hit the car'. That this is the \emph{correct} analysis can only be appreciated by bringing to
bear other aspects of the grammar.

\subsubsection{Syntactic categories}

We have recognised that sentences can be recursively analysed into `parts'. We have claimed that
each part belongs to a limited ranges of categories. In this section, we briefly examine those
categories.

First, we divide the class of categories in two: the \textbf{lexical categories} and the
\textbf{phrasal categories}. Every ultimate constituent (i.e. word) of the sentence belongs to a
lexical category. A higher-level structure that depends on a central word and which may contain more
than one word is a phrasal category.

\paragraph{Lexical categories}

The lexical categories consist of the `parts of speech'. Principal among these are: noun (N), verb
(V), adjective (Adj), adverb (Adv), preposition (Prep), determinative (D), subordinator,
coordinator, and interjection.

Thus, `a' is a determiner and `bird' is a noun. The syntax tree in Figure
\ref{fig:abirdhitthecar_lexical}, gives lexical-category labels for each of the words in `a bird hit
the car'. 

\begin{figure}[ht]
\Tree [.{} [.{} [.D a ] [.N bird ]  ] [.{} [.{} [.V hit ] ] [.{} [.D the ] [.N car ]]  ] ]
\caption{A syntax tree with lexical category labels}
\label{fig:abirdhitthecar_lexical}
\end{figure}

\paragraph{Phrasal categories}

A \textbf{phrase} is a constituent at a higher level than the lexical level. Each phrase typically
contains a `most important word' that determines its type. It may also contain other constituents
that serve to elaborate its meaning.

For example, a phrase consisting of a noun and supporting constituents is a \textbf{nominal}; a
nominal together with a determinative is a \textbf{noun phrase}; a verb with its supporting
constituents is a \textbf{verb phrase}; a noun phrase with a verb phrase is a \textbf{clause}.

The principal phrasal categories are: Clause (Clause), verb phrase (VP), noun phrase (NP),
nominal (N), adjective phrase (AdjP), adverb phrase (AdvP), preposition phrase (PP), and
determinative phrase (DP).

Figure \ref{fig:abirdhitthecar_phrasal} supplements our previous diagram with the inclusion of
phrasal category labels.

\begin{figure}[ht]
\Tree [.{Clause} [.{NP} [.D a ] [.N bird ]  ] [.{VP} [.{} [.V hit ] ] [.{NP} [.D the ] [.N car ]]  ] ]
\caption{A syntax tree with phrasal and lexical category labels}
\label{fig:abirdhitthecar_phrasal}
\end{figure}

\section{Grammatical function}

The last of our three foundational pillars of syntax is \textbf{grammatical function}. The function
of a constituent denotes its relationship either to the larger construction containing it to to
another element within that construction. The term `construction' is not defined in CGEL. I take it
to mean a phrase when considered as a structure whose elements are related to each other and which
may be related to higher-level structures.

The major two types of function are \emph{head} and \emph{dependent}. In the noun phrase `a bird',
the noun `bird' is the head and the determinative `a' is a dependent of the construction.

Each construction has at most one (and usually exactly one) \textbf{head}, its most important
element. Generally, the category of a phrase depends on that of the head: a phrase with a noun as a
head is a noun phrase, a phrase with a verb phrase as a head is a clause.

A phrase can contain more than one dependent. Consider the verb phrase `gave the key to the
landlord'. Its head is the verb `gave'. It has two dependents: `the key', a noun phrase and `to the
landlord', a preposition phrase.

Depending on the category of the construction at hand, we assign special names to the head and
dependents. For instance, a canonical clause has two constituents: a \textbf{subject} and a
\textbf{predicate}. Here, predicate is just a special case of head and subject is a special case of
dependent. The subject canonically takes the form of a noun phrase and the predicate a verb phrase.

We see this in Figure \ref{fig:abirdhitthecar_functional_phrasal} where `a bird' is the subject
(dependent) and `hit the car' is the predicate (head) of the clause. We use a triangle (or roof)
over a constituent to show that the analysis is not complete to the lexical level.

\begin{figure}[ht]
\Tree [.Clause \qroof{a bird}.{Subject: \\ NP}  \qroof{hit the car}.{Predicate: \\ VP} ]
\caption{A syntax tree with functional and phrasal labels}
\label{fig:abirdhitthecar_functional_phrasal}
\end{figure}

A phrase can have more than one dependent. In Figure \ref{fig:two_dependents_phrase}, we have a
syntax tree for the verb phrase `gave the key to the landlord'. The head of a verb phrase is called
a predicator and is a verb. Here, the verb `gave' is the head (predicator). Here the verb phrase has
two dependents: one an `object' and one a `complement'.

\begin{figure}[ht]
    \Tree [.VP [.{Predicator: \\ V} [.{} gave  ] ] \qroof{the key}.{Object: \\ NP}  
    \qroof{to the landlord}.{Complement: \\ PP} ]
\caption{A phrase with two dependents}
\label{fig:two_dependents_phrase}
\end{figure}

The previous example shows an important distinction between syntactic categories and grammatical
functions. Whereas the phrase `the key' is always a noun phrase in any context, its grammatical
function depends fundamental on the context. In the clause, `the key is shiny', the phrase `the key'
is a subject. But the same phrase is an object in the clause, `Bob stole the key'.

Not every construction has a head. The phrase `fish and chips' is, syntactically, a NP-coordination.
This is syntactic structure we have not yet seen so far. It is a NP and  also a
\textbf{coordination}, this being a relation between two or more elements of syntactically equal
status, the \textbf{coordinates}, usually linked by a \textbf{coordinator} such as \emph{and} or
\emph{or}.

Notably, `fish and chips' has no head. Instead it has two dependents, the coordinate `fish' and the
coordinate `and chips', each of which is a NP. The phrase `and chips' can be further analysed into
the marker `and' (a coordinator) and the coordinate `chips' (a NP). An example similar to this, `Kim
and Pat' is diagrammed in Chapter 11, Section 1.1 of CGEL.

\end{document}
